\documentclass[12pts,A4]{article}

\usepackage{amsmath}
\usepackage{amsfonts}


\begin{document}

\title{ECE108 Assignment 1}
\author{Yi Fan Yu (yf3yu@edu.uwaterloo.ca ) }
\date{Feb 22, 2017}
\maketitle


\section{Set Operation}




\begin{flushleft}
    
    \textbf{a)}
    $R \subseteq  S \iff R \subseteq  ((S - T) \cup (R \cap T))$ 


    This statement is false because the implication is only unidirectional. 
\bigskip



Proving $R \subseteq  S \rightarrow R \subseteq  ((S - T) \cup (R \cap T)) $ \\ 



$R \subseteq  ((S - T) \cup (R \cap T))$ can be simplified to using distributivity
\bigskip

$R \subseteq ( (( S - T )\cup R )\cap( ( S - T )\cup T) ) $ where


$( S - T )\cup R$ gives you a set X such that $R \subseteq X$

$ ( S - T )\cup T$  gives you S ...
\bigskip

	So... we get $ R \subseteq  (X \cap S) $ 
since $R \subseteq X$ and from our assumption $R \subseteq S$

The intersection gives at least R as an answer

Therefore
$R \subseteq  R$ is true 
 
\bigskip
The opposite way cannot be proved because R does'nt have to be  $\subseteq$ S
$R \subseteq  T$ will suffice the counter-example
\bigskip

Counter-example $ S = \{4\}$
$R = \{1,3,5\}$
$T = R = \{1,3,5\}$

$S-T = \{4\}$
$R \cap T = \{1,3,5\}$

$ R = (S - T) \cup (R \cap T) = \{1,3,4,5\}$ 
in this case R is definitely not a subset of S



\end{flushleft}

\pagebreak

\begin{flushleft}
    \textbf{(b)} $(A \cap C ) \subseteq (B \cap C) \rightarrow A \subseteq B$
    proof:
    \bigskip
    let $x \in A \cap B$
    we can conclude that... 
    \bigskip
    
    by the definition of intersection, $\forall x \in A, x \in C$
    
    we deduce that x must be in both A and C 

    \bigskip
    by the definition of subset, $\forall x \in (A \cap C), x \in (B \cap C)$

    
    we can say $\forall x \in A, x \in (B \cap C)$

    \bigskip
    which means by the definition of intersection that 

    $\forall x \in A, x \in B \wedge x\in C$

    So... $\forall x \in A, x$ must be $\in B$

    which is the definition of $ A \subseteq B $





\end{flushleft}


\begin{flushleft}
   \textbf{(c)} $A \in B \wedge B \in C \rightarrow A \in C$ 
    
    this statement is obviously false since

    let $ A = \{ 3 \}$
    let $ B = \{ \{ 3 \}, 4 \}$
    let $ C = \{ \{ \{ 3 \}, 4 \}, 5 \}$

    \bigskip

    we can see that $A \in B \wedge B \in C$

    but $A \not\in C$



\end{flushleft}


\begin{flushleft}
   \textbf{(d)} $A \in B \wedge B \in C \rightarrow A \subseteq C$ 
    
    this statement is obviously false since

    let $ A = \{ 3 \}$
    let $ B = \{ \{ 3 \}, 4 \}$
    let $ C = \{ \{ \{ 3 \}, 4 \}, 5 \}$

    \bigskip

    we can see that $A \in B \wedge B \in C$

    but $A \not\subseteq C$



\end{flushleft}



\begin{flushleft}
   \textbf{(e)} $A \in B \wedge B \subseteq C \rightarrow A \in C$ 
    
    proof: 

    if $B \subseteq C$

    that means $\forall x \in B , x \in C $
    

    now $A \in B$ means that $A$ is an element of B represented by $\forall x$

    replacing $\forall x$ by $A$
    
    we can then conclude $ A \in B$ means $ A \in C$
    
\end{flushleft}

\begin{flushleft}
    \textbf{(f)} $A \in B \wedge B \subseteq C \rightarrow A \subseteq C$ 
    
    this is obviously false since we proved that $A \in C$ is true
    let $ A = \{ 3 \}$
    let $ B = \{ \{ 3 \}, 4 \}$
    let $ C = \{ \{ 3 \}, 4 ,5 \}$



    \bigskip
    we can see that $A \in B \wedge B \subseteq C$

    but $A \not\subseteq C$




\end{flushleft}


\section{Set operations}


\begin{flushleft}
    Given sets A and B under what condition does $A-B= B-A$ 

    need to prove that $ A=B \iff A-B = B-A $
    

    \bigskip
    Proof:


    starting with $A=B \rightarrow A-B = B-A $
    
    if $A=B$, then $A-B = B-A = \O$
    
    
    \bigskip

    continuing with $A=B \leftarrow A-B = B-A$
    \bigskip
    Proof:
        
        the definitions for differences are:

        \smallskip
        $ A-B = \{x \mid x \in A \wedge x \not\in B\}$

        \smallskip

        $ B-A = \{x \mid x \in B \wedge x \not\in A\}$

        if $A-B = B-A$

        then it means that $\exists x \in A \mid x \not\in B$
        
        because of the equality,


        but the same x must exists in $B$ not in $A$

        we can see that no element 
        
        We can then conclude that

        $A - B = \O \wedge  B- A = \O$

        using the definition of difference we can see that

        in order for $A - B = \O$, that means that $A \subseteq B$
        
        in order for $B - A = \O$, that means that $B \subseteq A$
        
        \bigskip 
        therefore $A \subseteq B \wedge B \subseteq A$

        this is the definition of equality $A = B$.




        

\end{flushleft}


\section{Functions}

\begin{flushleft}
    \textbf{(a)}

    (i) if f is injective, we don't know anything about the relationship between co-domain and image
    we only know about that image $\subseteq$ co-domain

    (ii) image $=$ co-domain when it is surjective

    (iii) image $=$ co-domain when it is bijective 
    
     
\end{flushleft}

\begin{flushleft}
    \textbf{(b)}

    (i) if $f$ is injective, it means that the function is invertible and we can map the image ($\not =$ co-domain) back to it's domain


        so $f^{-1}$ has image the full codomain of $f$

        image  $(f^{-1} )$ = dom $(f)$
        

    (ii) turns out that if $f$ is not injective, it cannot be inverted
        
        since $f^{-1}$ does not exist, 
        
    (iii) image  $(f^{-1} )$ = dom $(f)$ when it is bijective 
    
    
     
\end{flushleft}

\section{Functions}

\begin{flushleft}
    
    \textbf{I assume that both X and Y are not empty sets....}


    \textbf{(a)} there exists an injection $f : X \rightarrow Y$
    \bigskip

    Proof:

    if $ X \subseteq Y$ , this means 
    
    $ \forall a ( a \in X \rightarrow a \in Y) $

    a "same" function can be applied that maps all the values in X to its same value, but in Y

    \bigskip
     $f : X \rightarrow Y$
    
     $ x \mapsto x$ 
    
    \bigskip

    Since all values in X is present in Y,
    
    and every single value in X maps to one value inside of Y
   
    and sets don't have duplicates

    we've got an injective "equivalence" function

    \bigskip
    \bigskip

    

    \textbf{(b)} there exists a surjection $g : Y \rightarrow X$

    well we know that $X \subseteq Y$

    so cardinality($X$) $\leq$ cardinality($Y$)
    
    I can conclude that my domain will be either equal or bigger than my codomain

    Therefore I can guarantee that my function will not be injective if I need my function to be surjective
    \bigskip

    so a function that 


    \begin{equation}
        x \mapsto \begin{cases}
            x,& \text{if $ x \in (X \cap Y)$}\\
            \text{any Value in Y}, & \text{if } x \in (Y - X)
        \end{cases}
    \end{equation}
    
%   $g : Y \rightarrow X$
    
%   $ x \mapsto x if x \in (X \cap Y)$ 
    
%   $and x \mapsto $ any value $ \in Y if x\in (Y - X)$
    \bigskip 
    the mapping to any value will make sure that our function definition maps all the domain to satisfy the definition of a function

\end{flushleft}


\section{ Functions.. Even more}

\begin{flushleft}
    \textbf{(a)} $f: \mathbb{N} \rightarrow \mathbb{N}$ where $ f: x \mapsto x$
    
    \bigskip 

    Since the dom $=$ codom, and the function maps the value to itself\\
    we can conclude that:
    
    \bigskip
    
    function is injective because all values of the domain is mapped to a unique value in the codomain 
    
    function is surjective because all values of the codomain is being mapped to
    (range $=$ codomain)

    therefore, the function is bijective

    \bigskip



    \textbf{(b)} $g: \mathbb{N} \rightarrow \mathbb{N}$ where $ g: x \mapsto x^{2}$

    in this case, the function maps all the values in the domain to the square of itself.

    since all values in the domain has a unique square value in the codomain, the function is injective

    since range $\not =$ codomain because $ 3 \in \mathbb{N}$ but 3 is not mapped by any value in the domain, the function is \textbf{NOT} surjective

    therefore not bijective 

    \bigskip

    \textbf{(c)} $h: \mathbb{Q^{+}} \rightarrow \mathbb{Q^{+}}$ where $ h: x \mapsto 1/x$
    
    \textbf{ I assume that the function maps x to the inverse of its most simplified version} else, it is not even a function 
    because $ 4 \mapsto \frac{1}{4}$ but also $\mapsto \frac{2}{8}$
    
    right off the bat, we can see that the same number could be represented by 2 different values of the domain i.e $ \frac{1}{2}$ $\frac{2}{4} $

    since the inverse operation of these 2 fractions all map to 2, the function does not satisfy the injective definition.

    similiarly, the surjective definition is not satisfied since $\frac{2}{8}$ will never be mapped to. refer to the assumption

    obviously not bijective, so \textbf{None of the Above}

\bigskip

    \textbf{(d)} possible to compose $ f \circ g $ $ f\circ h  $ $ g \circ h$ 
     
    in order to have a correct composition $ J \circ K$ also knowns as $ K(J(x)) $

    We know that codom(J) $\subseteq$ dom(K) since the domain of K can be restricted to $=$ codom(J)

    therefore:  


    cod(f) = $\mathbb{N}$  and dom(g) = $\mathbb{N}$ therefore possible 

    $g(f(x)): \mathbb{N} \rightarrow \mathbb{N}$ where $ x \mapsto x^{2}$


    cod(f) = $\mathbb{N}$  and dom(h) = $\mathbb{Q^{+}}$ therefore possible 
    
    $h(f(x)): \mathbb{N} \rightarrow \mathbb{Q^{+}}$ where $ x \mapsto \frac{1}{x}$

    cod(g) = $\mathbb{N}$  and dom(h) = $\mathbb{Q^{+}}$ therefore possible 
    
    $h(g(x)): \mathbb{N} \rightarrow \mathbb{Q^{+}}$ where $ x \mapsto \frac{1}{x^{2}}$
\end{flushleft}

\section{Closure}


\begin{flushleft}

    a strict partial order is asymmetric and transitive

    a partial order is reflective, antisymmetric and transitive

    \bigskip


    if we take the reflective closure of the relation $ < $ 
    (the new set is refered as $R$ from now on)

    then we have added all the $x<x$.\\

    More precisely: \\
    $ \forall x<y  , \exists (xRy \wedge xRx \wedge yRy)$

    where $ x \not = y$ since it is part of a strict poset 


    prove that transitivity is kept with the newly added elements:
    \bigskip

    as we see, for any arbitrary $xRy \in R$, we now have $xRx$ and $yRy$:

    using the definition of transitivity,
    
    $xRy \wedge yRy \rightarrow xRy$

    $xRx \wedge xRy \rightarrow xRy$
    
    then we see that $xRy$ is required to be in the set for it to be transitive

    and indeed $xRy$ is in the set from our assumption.
    \bigskip

    prove that the newly created set is antisymmetric:

    the new set R now satisfies the new condition 

    $ \forall x \forall y ((xRy \Rightarrow \neg yRx) \vee (x = y))$
    
    this means that for an arbitrary $xRy$, 
    there cannot be $yRx$ unless $ x = y$

    this is the less formal definition of antisymmetric relations
    
    \bigskip
    The proof for reflective is trivial since we had to take a reflective closure of the $<$ set.
    \bigskip

    Therefore, the new set $R$ is a poset since it satisfies the 3 conditions


    
\end{flushleft}

\section{Hass Diagram}


\section{Equivalent Relationship}

\begin{flushleft}

    need to show that T is reflective, symmetric and transitive
    

    \bigskip
    \bigskip

    we first need to determine the set relationship between $T$ and $R S$
    
    Define $T \subseteq A^{2}$ such that $xTy \iff (xRy \wedge xSy)$

    we see that for any arbitrary $xTy$ , there exists $xRy$ and $xSy$

    we can conclude that for any element in T, the same element exists in S and R

    mathematically, this is written as $T = R \subseteq T \wedge \subseteq S$
    
    which implies $\subseteq ( R \cap U)$ 
    \bigskip
    

    

    if $ T = \emptyset $, then $T$ would be an equivalent relationship since all the assumptions become false and implications become true
    
    so for our proof, we are going to assume that there exists at least 1 element inside the relation $T$

    
    \bigskip

    proof for reflective:
    
    \bigskip

    if T is not reflective, then $ \exists x   (\neg x T x) $

    so it means by double implication (iff) that $ \exists x( (\neg xRx)
    \vee (\neg xSx)) $

    

    but $R$ and $S$ are all equivalent relationships 

    Contraction occurs since both R and S are reflective and 

    $\forall x( xRx \wedge xSx)$

    we conclude that T has to be reflective

    \bigskip


    Same logic follows for the 2 other conditions:

    \bigskip

    proof for symmetric:
    
    \bigskip

    if T is not symmetric, then $ \exists x \exists y  ( xTy \wedge \neg y T x) $

    so it means by double implication (iff) that $ \exists x \exists y  ( xRy \wedge \neg y R x) 
    \vee ( xSy\wedge \neg y S x) $

    

    but $R$ and $S$ are all equivalent relationships 

    Contraction occurs since both R and S are symmetric  


    we conclude that T has to be symmetric


    \bigskip

    proof for transitive:
    
    \bigskip

    if T is not transitive, then $ \exists x \exists y \exists z
    (xTy \wedge yTz \wedge \neg xTz)$

    so it means by double implication (iff) that the same xSz or xRz doesn't exist
    
    $ \exists x \exists y \exists z( (xRy \wedge yRz \wedge \neg xRz) \vee
(xSy \wedge ySz \wedge \neg xSz))$
    

    

    but $R$ and $S$ are all equivalent relationships 

    Contraction occurs since both R and S are transitive, 
    so both $xRz$ and $xSz$ exists 


    we conclude that T has to be transitive


    \bigskip


    We finally conclude that T is an equivalent relationship... $\Box$



\end{flushleft}

\section{Posets}
\begin{flushleft}


\textbf{(a)} $ x \geq y \iff y \leq^{-1} x $ 

    $ x \geq y $ can be rewritten as
    $ y \leq x $

    since $y \leq x \not = y \leq^{-1} x$

    false
    \bigskip

\textbf{(b)} $ x \geq y \iff y \leq^{'} x $ 

    $ x \geq y $ can be rewritten as
    $ y \leq x $

    since $y \leq x \not = y \leq^{'} x$

    false

    \bigskip


\textbf{(c)} $ x < y \iff y \leq^{'} x $ 

    prove that $ x < y \Rightarrow y \leq^{'} x $  

    $x <  y$ can be rewritten as $ y >  x$

    and the complement of $ y > x$ is $ y \leq ^{'} x$ 
    
    therefore $y>x$ = $y \leq ^{'} x$
    \bigskip

    since I have proven they are equal, $\iff$ is proven

\textbf{(d)} $ x > y \iff y (\leq^{-1})^{'} x $ 

    $ x > y $ can be rewritten as $ y < x$

    $ y < x $ 's inverse is $ x < ^{-1} y $

    $ x < ^{-1} y $ can now be rewritten as $y > ^{-1} x$

    taking the complement might not be obvious, so let's split
    $X$ into $ \leq^{-1} ,>^{-1} $ sets
   
    taking the complement we get the set we don't have
    
    $y (\leq ^{-1})^{'} x$
    
    
    \bigskip

    we conclude that $ x > y =  y (\leq^{-1})^{'} x $  

    therefore the bidirection is proven since they are equal
    
    \bigskip

\textbf{(e)} $ x > y \iff y (\leq^{'})^{-1} x $ 
    
    I doubt this is true, since \textbf{(d)} is true...
    turns out it is true

    $ x > y $ can be rewritten as $ y < x$

    
    taking the complement we get  $ y < x$ = $ y \geq ^{'} x $
   
    if we then take the inverse we get $ x (\geq ^{'} )^{-1} y $
    
    indeed we just have to swap x and y to get the inverse
    
    this could be rewritten as $ y (\leq ^{'} )^{-1} x $
    \bigskip

    we conclude that $ x > y =  y (\leq^{'})^{-1} x $  

    therefore the bidirection is proven since they are equal


\end{flushleft}
\section {More Posets}

\section{posets until posets}

\section{Functions, Relations and Cardinality}

\begin{flushleft}

    to be a function every single value inside of A needs to map to so
    some value of the cod(A)

    let $A = \{ a,b,c \}$
    aa-ba-ca is an obvious one
    if a always maps to a,
    aa-ba-ca x3
    aa-ba-cb
    aa-ba-cc

    aa-bb-ca
    aa-bb-cb
    aa-bb-cc
    
    aa-bc-ca
    aa-bc-cb
    aa-bc-cc
    

    ab-ba-ca
    x9

    ac-ba-ca
    x9
    
    seems like the answer is $N^{N}$
    \bigskip


    Assuming A is not infinite,


    in order to get a surjective mapping, all values in the domain
    must map to something different since Dom = Cod indeed $A = A$

    this means that in order to get a surjective mapping, the function
    needs to be injective
    
    by being both injective and surjective, the function is bijective

    aa-bb-cc aa-bc-cb ab-ba-cc ab-bc-

    which is $N!$
\end{flushleft}

    i. Define $T \subseteq A2$ st $XTy \iff (xRy AND xSY)$
    show T is refl sym and transitive to prove it

4. Given poset (x, smallerEq) prove or disprove 
(a) x >= y iff y <=-1 x

R-1 = {(b,a) | (a,b) $\in$ R)
(y,x) $\in$ <= so it mean (x,y) $\in$ <=-1

(b) x >= y $\iff$ y <='x   (2,2) will prove it false;
\end{document}
