\documentclass[a4paper,12pt]{article}

\begin{document}

\title{ECE108 Assignment 1}
\author{Yi Fan Yu (yf3yu@edu.uwaterloo.ca ) }
\date{Feb 22, 2017}
\maketitle


\section{Set Operation}


a)
\begin{flushleft}

$R \subseteq  S \iff R \subseteq  ((S - T) \cup (R \cap T))$ 


This statement is false because the implication is only unidirectional. 
\bigskip



Proving $R \subseteq  S \rightarrow R \subseteq  ((S - T) \cup (R \cap T)) $ \\ 



$R \subseteq  ((S - T) \cup (R \cap T))$ can be simplified to using distributivity
\bigskip

$R \subseteq ( (( S - T )\cup R )\cap( ( S - T )\cup T) ) $ where


$( S - T )\cup R$ gives you a set X such that $R \subseteq X$

$ ( S - T )\cup T$  gives you S ...
\bigskip

	So... we get $ R \subseteq  (X \cap S) $ 
since $R \subseteq X$ and from our assumption $R \subseteq S$

The intersection gives at least R as an answer

Therefore
$R \subseteq  R$ is true 
 
\bigskip
The opposite way cannot be proved because R does'nt have to be  $\subseteq$ S

Counter-example $ S = \{4\}$
$R = \{1,3,5\}$
$T = R = \{1,3,5\}$

$S-T = \{4\}$
$R \cap T = \{1,3,5\}$

$ R = (S - T) \cup (R \cap T) = \{1,3,4,5\}$ 
in this case R is definitely not a subset of S



\end{flushleft}


\end{document}
