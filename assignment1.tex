\documentclass[a4paper,12pt]{article}

\begin{document}

\title{ECE108 Assignment 1}
\author{Yi Fan Yu (yf3yu@edu.uwaterloo.ca ) }
\date{Feb 22, 2017}
\maketitle


\section{Set Operation}


a)
\begin{flushleft}

$R \subseteq  S \iff R \subseteq  ((S - T) \cup (R \cap T))$ 


This statement is false because the implication is only unidirectional. 
\bigskip



Proving $R \subseteq  S \rightarrow R \subseteq  ((S - T) \cup (R \cap T)) $ \\ 



$R \subseteq  ((S - T) \cup (R \cap T))$ can be simplified to using distributivity
\bigskip

$R \subseteq ( (( S - T )\cup R )\cap( ( S - T )\cup T) ) $ where


$( S - T )\cup R$ gives you a set X such that $R \subseteq X$

$ ( S - T )\cup T$  gives you S ...
\bigskip

	So... we get $ R \subseteq  (X \cap S) $ 
since $R \subseteq X$ and from our assumption $R \subseteq S$

The intersection gives at least R as an answer

Therefore
$R \subseteq  R$ is true 
 
\bigskip
The opposite way cannot be proved because R does'nt have to be  $\subseteq$ S

Counter-example $ S = \{4\}$
$R = \{1,3,5\}$
$T = R = \{1,3,5\}$

$S-T = \{4\}$
$R \cap T = \{1,3,5\}$

$ R = (S - T) \cup (R \cap T) = \{1,3,4,5\}$ 
in this case R is definitely not a subset of S



\end{flushleft}



a) $R \subseteq  T$ proves it wrong then you are done

2.) Given sets A and B under what condition does $A-B= B-A$ 

    need to prove that $ A=B \iff A-B = B-A $
    

    \bigskip

    starting with $A=B \rightarrow A-B = B-A $
    
    if $A=B$, then $A-B = B-A = \O$
    
    
    \bigskip

    continuing with $A=B \leftarrow A-B = B-A$

    PBC    if $A\neq B$
        then $A\not\subseteq B \vee B\not\subseteq A$
        then $\exists x \in A \mid x \not\in B$ OR
            $\exists x \in B \mid x \not\in A$

    but by definition of difference of sets:\\
    
        $ A-B = \{x \mid x \in A \wedge x \not\in B\}$
        \smallskip
        
        
        $ B-A = \{x \mid x \in B \wedge x \not\in A\}$

        if $ A-B = B-A$

        then it means that $\exists x \in A \mid x \not\in B$
        
        because of the equality 
        but the same x must exists in $B$ not in $A$

        we can see that no element satisfies this condition
        
        We can then conclude that

        $A - B = \O$ $ B- A = \O$
        proving also that $\forall x \in A \mid x \in B$
                          $\forall x \in B \mid x \in A$
                          therefor proving the equality

        
       \bigskip 
        \bigskip


i. Define $T \subseteq A2$ st $XTy \iff (xRy AND xSY)$
    show T is refl sym and transitive to prove it

4. Given poset (x, smallerEq) prove or disprove 
(a) x >= y iff y <=-1 x

R-1 = {(b,a) | (a,b) $\in$ R)
(y,x) $\in$ <= so it mean (x,y) $\in$ <=-1

(b) x >= y $\iff$ y <='x   (2,2) will prove it false;
\end{document}
