\documentclass[12pts,A4]{article}

\usepackage{amsmath}
\usepackage{amsfonts}
\usepackage{tikz}

\begin{document}

\title{ECE108 Assignment 1}
\author{Yi Fan Yu (yf3yu@edu.uwaterloo.ca ) }
\date{Feb 22, 2017}
\maketitle


\section{Set Operation}




\begin{flushleft}

    \textbf{a)}
    $R \subseteq  S \iff R \subseteq  ((S - T) \cup (R \cap T))$ 

    This statement is false because the implication is only unidirectional. 
\bigskip


Proving $R \subseteq  S \rightarrow R \subseteq  ((S - T) \cup (R \cap T)) $ \\ 

$R \subseteq  ((S - T) \cup (R \cap T))$ can be simplified to using distributivity
\bigskip

$R \subseteq ( (( S - T )\cup R )\cap( ( S - T )\cup T) ) $ where

$( S - T )\cup R$ gives you a set X such that $R \subseteq X$

$ ( S - T )\cup T$  gives you S ...
\bigskip

	So... we get $ R \subseteq  (X \cap S) $ 
since $R \subseteq X$ and from our assumption $R \subseteq S$

The intersection gives at least R as an answer

Therefore
$R \subseteq  R$ is true 
 
\bigskip
The opposite way cannot be proved because R does'nt have to be  $\subseteq$ S
$R \subseteq  T$ will suffice the counter-example
\bigskip

Counter-example $ S = \{4\}$
$R = \{1,3,5\}$
$T = R = \{1,3,5\}$

$S-T = \{4\}$
$R \cap T = \{1,3,5\}$

$ R \subseteq (S - T) \cup (R \cap T) = \{1,3,4,5\}$ true
in this case R is definitely not a subset of S



\end{flushleft}

\pagebreak

\begin{flushleft}
    \textbf{(b)} $(A \cap C ) \subseteq (B \cap C) \rightarrow A \subseteq B$

    proof:
    \bigskip
    let $x \in A \cap B$
    we can conclude that... 
    \bigskip
    
    by the definition of intersection, $\forall x \in A, x \in C$
    
    we deduce that x must be in both A and C 

    \bigskip

    by the definition of subset, $\forall x \in (A \cap C), x \in (B \cap C)$

    
    we can say $\forall x \in A, x \in (B \cap C)$

    \bigskip
    which means by the definition of intersection that 

    $\forall x \in A, x \in B \wedge x\in C$

    So... $\forall x \in A, x$ must be $\in B$

    which is the definition of $ A \subseteq B $





\end{flushleft}


\begin{flushleft}
   \textbf{(c)} $A \in B \wedge B \in C \rightarrow A \in C$ 
    
    this statement is obviously false since

    let $ A = \{ 3 \}$
    let $ B = \{ \{ 3 \}, 4 \}$
    let $ C = \{ \{ \{ 3 \}, 4 \}, 5 \}$

    \bigskip

    we can see that $A \in B \wedge B \in C$

    but $A \not\in C$



\end{flushleft}


\begin{flushleft}
   \textbf{(d)} $A \in B \wedge B \in C \rightarrow A \subseteq C$ 
    
    this statement is obviously false since

    let $ A = \{ 3 \}$
    let $ B = \{ \{ 3 \}, 4 \}$
    let $ C = \{ \{ \{ 3 \}, 4 \}, 5 \}$

    \bigskip

    we can see that $A \in B \wedge B \in C$

    but $A \not\subseteq C$



\end{flushleft}



\begin{flushleft}
   \textbf{(e)} $A \in B \wedge B \subseteq C \rightarrow A \in C$ 
    
    proof: 

    if $B \subseteq C$

    that means $\forall x \in B , x \in C $
    

    now $A \in B$ means that $A$ is an element of B represented by $\forall x$

    replacing $\forall x$ by $A$
    
    we can then conclude $ A \in B$ means $ A \in C$
    
\end{flushleft}

\begin{flushleft}
    \textbf{(f)} $A \in B \wedge B \subseteq C \rightarrow A \subseteq C$ 
    
    this is obviously false since we proved that $A \in C$ is true
    let $ A = \{ 3 \}$
    let $ B = \{ \{ 3 \}, 4 \}$
    let $ C = \{ \{ 3 \}, 4 ,5 \}$



    \bigskip
    we can see that $A \in B \wedge B \subseteq C$

    but $A \not\subseteq C$




\end{flushleft}


\section{Set operations}


\begin{flushleft}
    Given sets A and B under what condition does $A-B= B-A$ 

    need to prove that $ A=B \iff A-B = B-A $
    

    \bigskip
    Proof:


    starting with $A=B \rightarrow A-B = B-A $
    
    if $A=B$, then $A-B = B-A = \O$
    
    
    \bigskip

    continuing with $A=B \leftarrow A-B = B-A$
    \bigskip
    Proof:
        
        the definitions for differences are:

        \smallskip
        $ A-B = \{x \mid x \in A \wedge x \not\in B\}$

        \smallskip

        $ B-A = \{x \mid x \in B \wedge x \not\in A\}$

        if $A-B = B-A$

        then it means that $\exists x \in A \mid x \not\in B$
        
        because of the equality,


        but the same x must exists in $B$ not in $A$

        we can see that no element 
        
        We can then conclude that

        $A - B = \O \wedge  B- A = \O$

        using the definition of difference we can see that

        in order for $A - B = \O$, that means that $A \subseteq B$
        
        in order for $B - A = \O$, that means that $B \subseteq A$
        
        \bigskip 
        therefore $A \subseteq B \wedge B \subseteq A$

        this is the definition of equality $A = B$.




        

\end{flushleft}


\section{Functions}

\begin{flushleft}
    \textbf{(a)}

    (i) if f is injective, we don't know anything about the relation between co-domain and image
    ok i lied... \\ 
    we only know about that image $\subseteq$ co-domain\\
    
    (ii) image $=$ co-domain when it is surjective

    (iii) image $=$ co-domain when it is bijective 
    
     
\end{flushleft}

\begin{flushleft}
    \textbf{(b)}

    (i) if $f$ is injective, it means that the function is invertible and we can map the image ($\not =$ co-domain) back to it's domain


        so $f^{-1}$ has image the full codomain of $f$

        image  $(f^{-1} )$ = dom $(f)$
        

    (ii) turns out that if $f$ is not injective, it cannot be inverted
        
        since $f^{-1}$ does not exist, 
        
    (iii) image  $(f^{-1} )$ = dom $(f)$ when it is bijective 
    
    
     
\end{flushleft}

\section{Functions}

\begin{flushleft}
    
    \textbf{I assume that both X and Y are not empty sets....}


    \textbf{(a)} there exists an injection $f : X \rightarrow Y$
    \bigskip

    Proof:

    if $ X \subseteq Y$ , this means 
    
    $ \forall a ( a \in X \rightarrow a \in Y) $

    a "same" function can be applied that maps all the values in X to its same value, but in Y

    \bigskip
     $f : X \rightarrow Y$
    
     $ x \mapsto x$ 
    
    \bigskip

    Since all values in X is present in Y,
    
    and every single value in X maps to one value inside of Y
   
    and sets don't have duplicates

    we've got an injective "equivalence" function

    \bigskip
    \bigskip

    

    \textbf{(b)} there exists a surjection $g : Y \rightarrow X$

    \textbf{this is false if I didn't assume non-empty sets since if X = $\emptyset$, it is no longer a function}

    \bigskip

    \bigskip
    \textbf{else,}

    well we know that $X \subseteq Y$

    so cardinality($X$) $\leq$ cardinality($Y$)
    
    I can conclude that my domain will be either equal or bigger than my codomain

    Therefore I can guarantee that my function will not be injective if my function is surjective
    \bigskip

    so  I can have a  function that 


    \begin{equation}
        x \mapsto \begin{cases}
            x,& \text{if $ x \in (X \cap Y)$}\\
            \text{any Value in Y}, & \text{if } x \in (Y - X)
        \end{cases}
    \end{equation}
   \bigskip 
    the mapping to any value will make sure that our function definition maps all the domain to satisfy the definition of a function

\end{flushleft}


\section{ Functions.. Even more}

\begin{flushleft}
    \textbf{(a)} $f: \mathbb{N} \rightarrow \mathbb{N}$ where $ f: x \mapsto x$
    
    \bigskip 

    Since the dom $=$ codom, and the function maps the value to itself\\
    we can conclude that:
    
    \bigskip
    
    function is injective because all values of the domain is mapped to a unique value in the codomain 
    
    function is surjective because all values of the codomain is being mapped to
    (range $=$ codomain)

    therefore, the function is bijective

    \bigskip



    \textbf{(b)} $g: \mathbb{N} \rightarrow \mathbb{N}$ where $ g: x \mapsto x^{2}$

    in this case, the function maps all the values in the domain to the square of itself.

    since all values in the domain has a unique square value in the codomain, the function is injective

    since range $\not =$ codomain because $ 3 \in \mathbb{N}$ but 3 is not mapped by any value in the domain, the function is \textbf{NOT} surjective

    therefore not bijective 

    \bigskip

    \textbf{(c)} $h: \mathbb{Q^{+}} \rightarrow \mathbb{Q^{+}}$ where $ h: x \mapsto 1/x$
    
    \textbf{ I assume that the function maps x to the inverse of its unsimplified version} 
    
    so the function $h$ can be described as doing this:

    $\forall_{x y} \in \mathbb{N} \frac{x}{y} \mapsto \frac{y}{x}$
    \bigskip

    we can then say that every value in the codomain has its correponding value in the domain

    since we can map all $\frac{x}{y}$ to an unique $\frac{y}{x}$, it is injective

    \bigskip
    since we can represent all values of $\mathbb{Q^{+}}$ with $y$ and $x \in \mathbb{N}$ and that there is no value that cannot be mapped by the domain, it is surjective

    \bigskip

    therefore $h$ is bijective too

    \bigskip
    \textbf{(d)} possible to compose $ f \circ g $ $ f\circ h  $ $ g \circ h$ 
     
    in order to have a correct composition $ J \circ K$ also knowns as $ K(J(x)) $

    We know that codom(J) $\subseteq$ dom(K) since the domain of K can be restricted to $=$ codom(J)

    therefore:  


    cod(f) = $\mathbb{N}$  and dom(g) = $\mathbb{N}$ therefore possible 

    $g(f(x)): \mathbb{N} \rightarrow \mathbb{N}$ where $ x \mapsto x^{2}$


    cod(f) = $\mathbb{N}$  and dom(h) = $\mathbb{Q^{+}}$ therefore possible 
    
    $h(f(x)): \mathbb{N} \rightarrow \mathbb{Q^{+}}$ where $ x \mapsto \frac{1}{x}$

    cod(g) = $\mathbb{N}$  and dom(h) = $\mathbb{Q^{+}}$ therefore possible 
    
    $h(g(x)): \mathbb{N} \rightarrow \mathbb{Q^{+}}$ where $ x \mapsto \frac{1}{x^{2}}$
\end{flushleft}

\section{Closure}


\begin{flushleft}

    a strict partial order is asymmetric and transitive

    a partial order is reflective, antisymmetric and transitive

    \bigskip


    if we take the reflective closure of the relation $ < $ 
    (the new set is refered as $R$ from now on)

    then we have added all the $xRx$.\\

    More precisely: \\
    $ \forall x<y  , \exists (xRy \wedge xRx \wedge yRy)$

    where $ x \not = y$ since it is part of a strict poset 


    prove that transitivity is kept with the newly added elements:
    \bigskip

    as we see, for any arbitrary $xRy \in R$, we now have $xRx$ and $yRy$:

    using the definition of transitivity,
    
    $xRy \wedge yRy \rightarrow xRy$

    $xRx \wedge xRy \rightarrow xRy$
    
    then we see that $xRy$ is required to be in the set for it to be transitive

    and indeed $xRy$ is in the set from our assumption.
    \bigskip

    prove that the newly created set is antisymmetric:

    the new set R now satisfies the new condition 

    $ \forall x \forall y ((xRy \Rightarrow \neg yRx) \vee (x = y))$
    
    this means that for an arbitrary $xRy$, 
    there cannot be $yRx$ unless $ x = y$

    this is the less formal definition of antisymmetric relations
    
    \bigskip
    The proof for reflective is trivial since we had to take a reflective closure of the $<$ set.
    \bigskip

    Therefore, the new set $R$ is a poset since it satisfies the 3 conditions


    
\end{flushleft}

\section{Hass Diagram}


\begin{flushleft}
    $INC \geq F $actually $INC > F$\\
    $INC \geq D$\\
    $A > INC$\\

    $WD > F$ \\
    $WD > D$ \\
    $B \geq WD $\\
    $AEG > F$\\
    $AEG > D$\\
    $AEG \geq C$\\
    
    \pagebreak

    \textbf{ better relation} only transitive 

    \begin{tikzpicture}[scale=2,thick ]
      \node (a) at (0,-3) {$A$};
      \node (b) at (0,-2) {$B$};
      \node (c) at (0,-1) {$C$};
      \node (d) at (0,0) {$D$};  
      \node (f) at (0,1) {$F$};
      \node (inc) at (-2,-1) {$INC$};
      \node (wd) at (2,-1) {$WD$};
      \node (aeg) at (-1,-1) {$AEG$};
        \draw[ -latex ] (a) edge (b)  (b) edge (c)  (c) edge (d)
        (d) edge (f) (aeg) edge (d)  (wd) edge (d)
        (a) edge (inc) (inc) edge (f) ;
    \end{tikzpicture}
   
    \textbf{ better or equal relation} transitive and reflective 

    \begin{tikzpicture}[scale=2,thick ]
      \node (a) at (0,-3) {$A$};
      \node (b) at (0,-2) {$B$};
      \node (c) at (0,-1) {$C$};
      \node (d) at (0,0) {$D$};  
      \node (f) at (0,1) {$F$};
      \node (inc) at (-2,-1) {$INC$};
      \node (wd) at (2,-1) {$WD$};
      \node (aeg) at (-1,-1) {$AEG$};
        \draw[ -latex ] (a) edge (b)  (b) edge (c)  (c) edge (d)
        (d) edge (f) (aeg) edge (c) (b) edge (wd) (wd) edge (d)
        (a) edge (inc) (inc) edge (d) ;
    \end{tikzpicture}
    

    \textbf{(c)}
    
    so better than is not a poset because it is not reflective, therefore I cannot find GLD and LUB

    $GLB$ for better or equal relation does not exist because $A$ is not related to $AEG$
    
    therefore not satisfying the condition to be a lower bound


    $LUB$ for better or equal relation is $F$ since $\forall x \in G xRF$

    \textbf{(d)}
    
    Again, only considering the better than equal relation since it is a poset and maximal and minimal elements apply on posets

    over the $G - \{A, F\} $ 

    \textbf{I will call the poset relation $\leq$ so i don't confuse myself with the definitions of minimal and maximal}
    
    the maximal element is $D$, since $\not\exists y$ in the subset $>$ than D

    the minimal element are $B, INC, AEG$ since they are not related to each other, but $\not\exists y$ in the subset $<$ than these values.
    
    \bigskip

    \bigskip

    \textbf{(e)}

    only the better than equal relation is a poset, since it is reflective (equal), transitive, and antisymmetric and can be shown with a Hass Diagram

    the better than relation is not a poset because it is irreflective by definition\\
    \bigskip
    the assumption we make is that $INC$ $AEG$ and $WD$ have no relation with each other in order for the "better than equal relation\\
    to qualify as as a partial order
    
\end{flushleft}

\section{Equivalent Relationship}

\begin{flushleft}

    need to show that T is reflective, symmetric and transitive
    

    \bigskip
    \bigskip

    we first need to determine the set relation between $T$ and $R S$
    
    Define $T \subseteq A^{2}$ such that $xTy \iff (xRy \wedge xSy)$

    we see that for any arbitrary $xTy$ , there exists $xRy$ and $xSy$

    we can conclude that for any element in T, the same element exists in S and R

    mathematically, this is written as $T = R \subseteq T \wedge \subseteq S$
    
    which implies $\subseteq ( R \cap U)$ 
    \bigskip
    

    

    if $ T = \emptyset $, then $T$ would be an equivalent relation since all the assumptions become false and implications become true
    
    so for our proof, we are going to assume that there exists at least 1 element inside the relation $T$

    
    \bigskip

    proof for reflective:
    
    \bigskip

    if T is not reflective, then $ \exists x   (\neg x T x) $

    so it means by double implication (iff) that $ \exists x( (\neg xRx)
    \vee (\neg xSx)) $

    

    but $R$ and $S$ are all equivalent relations 

    Contraction occurs since both R and S are reflective and 

    $\forall x( xRx \wedge xSx)$

    we conclude that T has to be reflective

    \bigskip


    Same logic follows for the 2 other conditions:

    \bigskip

    proof for symmetric:
    
    \bigskip

    if T is not symmetric, then $ \exists x \exists y  ( xTy \wedge \neg y T x) $

    so it means by double implication (iff) that $ \exists x \exists y  ( xRy \wedge \neg y R x) 
    \vee ( xSy\wedge \neg y S x) $

    

    but $R$ and $S$ are all equivalent relations 

    Contraction occurs since both R and S are symmetric  


    we conclude that T has to be symmetric


    \bigskip

    proof for transitive:
    
    \bigskip

    if T is not transitive, then $ \exists x \exists y \exists z
    (xTy \wedge yTz \wedge \neg xTz)$

    so it means by double implication (iff) that the same xSz or xRz doesn't exist
    
    $ \exists x \exists y \exists z( (xRy \wedge yRz \wedge \neg xRz) \vee
(xSy \wedge ySz \wedge \neg xSz))$
    

    

    but $R$ and $S$ are all equivalent relations 

    Contraction occurs since both R and S are transitive, 
    so both $xRz$ and $xSz$ exists 


    we conclude that T has to be transitive


    \bigskip


    We finally conclude that T is an equivalent relation... $\Box$



\end{flushleft}

\section{Posets}
\begin{flushleft}


\textbf{(a)} $ x \geq y \iff y \leq^{-1} x $ 

    $ x \geq y $ can be rewritten as
    $ y \leq x $

    since $y \leq x \not = y \leq^{-1} x$

    for when $y \not = x$

    false
    \bigskip

\textbf{(b)} $ x \geq y \iff y \leq^{'} x $ 

    $ x \geq y $ can be rewritten as
    $ y \leq x $

    since $y \leq x \not = y \leq^{'} x$
    
    because $y \leq^{'} x = y > x $ by definition of complement

    false

    \bigskip


\textbf{(c)} $ x < y \iff y \leq^{'} x $ 

    prove that $ x < y \Rightarrow y \leq^{'} x $  

    $x <  y$ can be rewritten as $ y >  x$

    and the complement of $ y > x$ is $ y \leq ^{'} x$ 
    
    therefore $y>x$ = $y \leq ^{'} x$
    \bigskip

    since I have proven they are equal, $\iff$ is proven

\textbf{(d)} $ x > y \iff y (\leq^{-1})^{'} x $ 

    $ x > y $ can be rewritten as $ y < x$

    $ y < x $ 's inverse is $ x < ^{-1} y $

    $ x < ^{-1} y $ can now be rewritten as $y > ^{-1} x$

    taking the complement might not be obvious, so let's split
    $X$ into $ \leq^{-1} ,>^{-1} $ sets
   
    taking the complement we get the set we don't have
    
    $y (\leq ^{-1})^{'} x$
    
    
    \bigskip

    we conclude that $ x > y =  y (\leq^{-1})^{'} x $  

    therefore the bidirection is proven since they are equal
    
    \bigskip

\textbf{(e)} $ x > y \iff y (\leq^{'})^{-1} x $ 
    
    I doubt this is true, since \textbf{(d)} is true...
    turns out it is true


    
    taking the complement we get  $ x > y$ = $ x \leq ^{'} y $
   
    if we then take the inverse we get $ y (\leq ^{'} )^{-1} x $
    
    indeed we just have to swap x and y to get the inverse
    
    \bigskip

    we conclude that $ x > y =  y (\leq^{'})^{-1} x $  

    therefore the bidirection is proven since they are equal


\end{flushleft}
\section {More Posets}

\begin{flushleft}
    
    $ X \subseteq N $

    $ \forall x,y \in \mathbb{N}\, xRy \iff \exists_{z \in X} \,x+z = y$

    \textbf{(a)} $0 \in X$

    let us assume that set $X$ has at least one element $a$ 
    
    therefore, by the $aRa$ must exist since R is a poset and has to be symmetric
    that implies that there must exist in $X$ to replace $z$ that will make the equation
    
    $ x+z = y$\\
    with $ x=a$ and $y=a$

    $a + z = a $\\
    we conclude that z must be $=0$ and thus $ 0 \in X$


    
    
    \textbf{(b)} 
    I modify the question $ a=x$ and $ b=y$ so it becomes 
    $$ \forall_{a b} (a \in X \wedge b \in X) \Rightarrow a + b \in X $$
    we know from part \textbf{(a)}  and assumption

    
    $0,a,b \in X$

    where $a$ and $b$ are just arbitrary values $\in X$
    
    we then know that because R is a poset (symmetrical)

    $ 0R0 , aRa, bRb $ exists in the poset
    
    \bigskip
    we are trying to prove that $a+b \in X$
    
    let $ x = y = (a+b) $\\
    we know that $ (x, y)\in \mathbb{N}$\\
    so we need to show that $(a+b) \in \mathbb{N}$\\
    since $ (a \wedge b) \in X \wedge X \subseteq \mathbb{N}$\\

    we can deduce that $(a \wedge b) \in \mathbb{N}$ by applying what we found in \textbf{question 1}\\ 
    therefore since both $a$ and $b$ are $ \in \mathbb{N}$\\
    because addition is closed under $\mathbb{N}$\\ 
    we conclude that the substituion is possible and $a+b \in \mathbb{N}$\\

    \bigskip
    therefore by setting $z = 0$ since $0 \in X$\\

    we get $(a+b) + 0 = (a+b)$ which is true

    this then implies by our assumption ( $\iff$ statement) 
    that $ (a+b)R(a+b)$ exists, which implies that $a+b$ is part of the base set since the poset is reflective\\
    (there must be $xRx \forall x \in X$)

\end{flushleft}

\section{posets until posets}
\begin{flushleft}

        
    \textbf{(a)}
    Proof that minimum element is unique

    PBC let's assume $a$ and $b$ are the minimum elements of the subset $Y$

    then by the $\iff$ statement,
    
    $\forall y \in Y a \leq y $ (eq1) 

    $\forall y \in Y b \leq y $ 
    
    and both $ a \wedge b \in Y$

    there is a contradiction here since we have $\forall y \in Y$
    
    $y$ can take the value of $ a$ and $b$

    so it must be true that by manipulating (eq1) that
    
    $ b \in Y a \leq b $

    which is saying that a is $\leq$ b
    but b cannot be smaller than any value because b is a minimum 

    \bigskip

    therefore a minimum element is unique $\Box$
    
    \bigskip

    \textbf{(b)}

    Proof that  minimum element $\Rightarrow$   minimal element
    
    \bigskip

    let a be the minimum element 

    then we know that $ a \in Y \wedge \forall_{b \in Y} \, a \leq b $ 
   
    PBC minimum element $\wedge \neg$ minimal element

    so if $a$ is not a minimal element, then $\exists_{z \in Y} \, z < a$

    in plain english, there exists a value $z \in Y$ such that $z$ is $<$ than $a$

    but will know from the definition of minimum element that all values of set Y is $\geq a$\\
    therefore contradiction occurs\\ 
    \bigskip
    minimum element $\Rightarrow$ minimal element $\Box$
    
    \bigskip

    \textbf{(c)}

    this is false,

    consider the poset  $ R = \{ (a,b),(c,b) (b,d) \}^{tran\,\, refl} $
    
    \begin{tikzpicture}[scale=.7]
      \node (d) at (0,3) {$d$};
      \node (b) at (0,1) {$b$};
      \node (c) at (2,0) {$c$};
      \node (a) at (-2,0) {$a$};  
        \draw[->] (a) edge (b)  (c) edge (b)  (b) edge (d);
    \end{tikzpicture}


    then let $ Y = R$ 
    
    the minimal element is $ a$ and $c$\\
    the minimum element doesn't exist since $a$ is not related to $c$\\
    $\forall_{y \in Y} \, x\leq y$ is not satisfied

    \bigskip
    
    we see that minimal element doesn't have to be the minimum element
    

    \bigskip
    \textbf{(d)}
    
    since totally ordered means connected partial order\\
    $ \forall_{x}\forall_{y} (xRy \vee yRx \vee x=y)$
    
    We claim that there exists a minimum no matter which subset of the base set X we take. 
    
    This means every single element inside the base set will become at least once a minimum element 
    
    Lets start with a subset $Y = X$ with minimum value $a$
    then by the definition of minimum, a interacts with the entire set by being $ \leq Y$ 
    $a$ is connected all the values in $X$

    \bigskip
    we then remove $a$ from $Y$ to create a set $Y^{'} \subseteq X$

    let's say b is now the minimum, so b is connected to every element in $X$ except for the element $a$ , but $a$ is already connected to $b$ previously


    \bigskip
    
    we then remove $b$ from $Y^{'}$ to create a new subset $Y^{''}$\\
    we don't have to worry about $a$ and $b$ since both of them are connected\\

    \bigskip

    now let's say $c$ is the new minimum, then $c$ is not connected to every single elements in $X$ including $a$ and $b$ done previously.

    If we repeat this recursive process, we will find that all the elements are connected since every time we remove a minimum from the subset, the new minimum element is connected to all the values in $X$.
    
    \bigskip

    This is the proof that it is a total order 
     


    \end{flushleft} 

\section{Functions, Relations and Cardinality}

\begin{flushleft}
    \textbf{(a)}


    to be a function every single value inside of A needs to map to so
    some value of the cod(A)
k
    let $A = \{ a,b,c \}$
    aa-ba-ca is an obvious one
    if a always maps to a,
    aa-ba-ca x3
    aa-ba-cb
    aa-ba-cc

    aa-bb-ca
    aa-bb-cb
    aa-bb-cc
    
    aa-bc-ca
    aa-bc-cb
    aa-bc-cc
    

    ab-ba-ca
    x9

    ac-ba-ca
    x9
    
    seems like the answer is $N^{N}$
    \bigskip

    \textbf{(b)}

    Assuming A is not infinite,


    in order to get a surjective mapping, all values in the domain
    must map to something different since Dom = Cod indeed $A = A$

    this means that in order to get a surjective mapping, the function
    needs to be injective
    
    by being both injective and surjective, the function is bijective
    
    possible bijective functions for 3 elements

    aa-bb-cc aa-bc-cb ab-ba-cc ab-bc-ca ac-ba-cb ac-bb-ca 

    by going through them one by one, 
    
    I conclude that the amount of bijective relation is $N!$

    therefore the amount of injective and surjective is also $N!$

    \bigskip
    
    \textbf{(c)}
    
    well let's brute force through all the possible relations:

    if 0 element $\rightarrow \{\} \binom{0}{0}$

    if 1 element $\rightarrow \{\} , \{a,a\} \binom{1} {0 to 1} $ the size is 2

    if 2 elements we get $ \emptyset , aa , ab, ba, bb, $ so $ \binom{4} {0 to 4} $ the size is 16

    if 3 elements we get $ \emptyset, aa , ab, ac, ba, bb, bc, ca, cb, cc $ so $ \binom{9} {0 to 9}$ the size is 512

    so we can conclude in general we get 
    $$\sum_{k=0}^{N^{2}} \binom{N^{2}} {k} $$ 

    is there a way to simplify?

    ya... $$ 2^{N^2} $$



    \textbf{(d)}
    
    to be both antisym and sym it means that the function needs to satisfy

    $ \forall x \forall y ((xRy \Rightarrow yRx) \wedge ((xRy \wedge yRx) \Rightarrow x = y ))$ 

    this basically means that only $xRx$ relation can exist inside of the set;
    
    attempt to prove:

    since if $xRy$ exists, then $yRx$ exists by symmetric definition
    
    but by the antisymmetric definition, $xRy \wedge yRx \Rightarrow  x = y$

    therefore $xRy = yRx$ therefore

    $ R = \{(x,x)\} $

    \bigskip

    by going through the similar process we realize its the sum of N choose K
    if 0 element $\rightarrow \{\} \binom{0}{0}$

    if 1 element $\rightarrow \{\} , \{a,a\} \binom{1} {0 to 1} $ the size is 2

    if 2 elements we get $ \emptyset , aa ,  bb, $ so $ \binom{2} {0 to 2} $ 

    $$\sum_{k=0}^{N} \binom{N} {k} $$ 
   
    simplified to $ 2 ^{N}$
    
    \bigskip

    \textbf{(e)}
    
    now $ R = \{  \forall x \in A ( xRx ) \} $ 

    if 0 element $\rightarrow \{\}$

    if 1 element $\rightarrow  \{(a,a)\}$ 

    if 2 elements we get , $\{ (a,a),(b,b) \}$

    we see that since it's for all x, the amount of set is 1 for all values of N 

   \bigskip

   \pagebreak

    \textbf{(f)}
    

    it is now reflective, symmetric, transitive
    
    then if f is referring to e, the amount of set is 1 for all $N$ because all set in \textbf{e} is transitive

    \textbf{the next page is just going to be me struggling to find the solution if it is refering to all possible relations}
    
    \pagebreak 
    
e   else it is very challenging... \\
    
    if 0 element $\rightarrow \{\}$

    if 1 element $\rightarrow \{\} , \{(a,a)\}$ 

    if 2 elements we get $ \emptyset , \{ (a,a),(b,b) \}$
                         $ aa-bb-ab-ba$

    if 3 elements we get $\emptyset, aa-bb-cc$

    also we get aa-bb-cc-ab-ba seems like 3 choose 2
                aa-bb-cc-ac-ca
                aa-bb-cc-cb-bc
                aa-bb-cc-ab-ba-ac-ca-cb-bc

    if 4 elements we get\\
                
                aa-bb-cc-dd\\

                aa-bb-cc-dd\\
                aa-bb-cc-dd-ab-ba with ab\\
                aa-bb-cc-dd-ac-ca with ac\\
                aa-bb-cc-dd-ad-da with ad\\
                aa-bb-cc-dd-bc-cb with cb\\
                aa-bb-cc-dd-bd-db with bd\\
                aa-bb-cc-dd-cd-dc with cd\\

                $\binom {4}{2}$

                aa-bb-cc-dd-ab-ba-bc-cb

                aa-bb-cc-dd-ac-ca-bd-db

                aa-bb-cc-dd-ad-da-bc-cb

                for num = 2 this step gives you 0\\
                for num = 3 this step gives you 1\\
                for num = 4 this step gives you 3\\
                for num = 5 this step gives you 8\\

                I ve come to a conclusion that it is $(N-3)*(N-1)$
                aa-bb-cc-dd-ab-ba-bc-cb-ac-ca \\
                aa-bb-cc-dd-bc-cb-cd-dc-bd-db with acd\\
   
   \bigskip
   it seems to me that it is $ 2 $ and $\binom {N} {1}$
   then it looks like it is $\binom {N}{2} $ where the the 2 elements from the base set must not be adjacent to each other 
    because if we take ab ba and 
    
\end{flushleft}
\end{document}
